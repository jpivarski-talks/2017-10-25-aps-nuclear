\documentclass[aspectratio=169]{beamer}

\mode<presentation>
{
  \usetheme{default}
  \usecolortheme{default}
  \usefonttheme{default}
  \setbeamertemplate{navigation symbols}{}
  \setbeamertemplate{caption}[numbered]
  \setbeamertemplate{footline}[frame number]  % or "page number"
  \setbeamercolor{frametitle}{fg=white}
  \setbeamercolor{footline}{fg=black}
} 

\usepackage[english]{babel}
\usepackage[utf8x]{inputenc}
\usepackage{tikz}
\usepackage{courier}
\usepackage{array}
\usepackage{bold-extra}
\usepackage{minted}
\usepackage[thicklines]{cancel}

\xdefinecolor{dianablue}{rgb}{0.18,0.24,0.31}
\xdefinecolor{darkblue}{rgb}{0.1,0.1,0.7}
\xdefinecolor{darkgreen}{rgb}{0,0.5,0}
\xdefinecolor{darkgrey}{rgb}{0.35,0.35,0.35}
\xdefinecolor{darkorange}{rgb}{0.8,0.5,0}
\xdefinecolor{darkred}{rgb}{0.7,0,0}
\definecolor{darkgreen}{rgb}{0,0.6,0}
\definecolor{mauve}{rgb}{0.58,0,0.82}

\title[2017-10-25-aps-nuclear]{Bridging the Particle Physics and Big Data Worlds}
\author{Jim Pivarski}
\institute{Princeton University -- DIANA-HEP}
\date{October 25, 2017}

\begin{document}

\logo{\pgfputat{\pgfxy(0.11, 7.4)}{\pgfbox[right,base]{\tikz{\filldraw[fill=dianablue, draw=none] (0 cm, 0 cm) rectangle (50 cm, 1 cm);}\mbox{\hspace{-8 cm}\includegraphics[height=1 cm]{princeton-logo-long.png}\includegraphics[height=1 cm]{diana-hep-logo-long.png}}}}}

\begin{frame}
  \titlepage
\end{frame}

\logo{\pgfputat{\pgfxy(0.11, 7.4)}{\pgfbox[right,base]{\tikz{\filldraw[fill=dianablue, draw=none] (0 cm, 0 cm) rectangle (50 cm, 1 cm);}\mbox{\hspace{-8 cm}\includegraphics[height=1 cm]{princeton-logo.png}\includegraphics[height=1 cm]{diana-hep-logo.png}}}}}

% Uncomment these lines for an automatically generated outline.
%\begin{frame}{Outline}
%  \tableofcontents
%\end{frame}

%%%%%%%%%%%%%%%%%%%%%%%%%%%%%%%%%%%%%%%%%%%%%%%%%%%%%%%

%%%% START

\begin{frame}{Big Data}
\vspace{0.5 cm}

\large For decades, physicists' computing needs were almost unique:

\vspace{0.1 cm}
\begin{itemize}
\item big datasets \hfill \begin{minipage}{0.7\linewidth}(too large for one computer: a moving definition!),\end{minipage}
\item complex structure \hfill \begin{minipage}{0.7\linewidth}(nested data, web of relationships within each event),\end{minipage}
\item has to be reduced \hfill \begin{minipage}{0.7\linewidth}(aggregated, by histogramming, usually)\end{minipage}
\item to be modeled \hfill \begin{minipage}{0.7\linewidth}(fitting to extract physics results).\end{minipage}
\end{itemize}

\vspace{0.5 cm}
\uncover<2->{\large Today these criteria apply equally, or more so, to ``web scale data.''}
\end{frame}

\begin{frame}{{200~PB is a lot of data}\only<2>{, but for Amazon, it's two trucks}}
\vspace{0.35 cm}
\includegraphics[width=0.73\linewidth]{cern-200pb.png}

\vspace{-4.8 cm}
\uncover<2->{\mbox{ } \hfill \includegraphics[width=0.7\linewidth]{aws-snowmobile.jpg}\hspace{-1 cm}}
\end{frame}

\begin{frame}{Bigger data}
\vspace{0.5 cm}

\large In fact, they're solving {\it harder} problems than ours

\vspace{0.1 cm}
\begin{itemize}
\item all-to-all problems \hfill \begin{minipage}{0.7\linewidth}(relationships span whole dataset: e.g.\ market basket),\end{minipage}
\item live updates \hfill \begin{minipage}{0.7\linewidth}(dataset grows while you analyze it),\end{minipage}
\item no ansatz \hfill \begin{minipage}{0.7\linewidth}(modeling human behavior, often predictively),\end{minipage}
\item complex models \hfill \begin{minipage}{0.7\linewidth}(thousands of free parameters; machine learning).\end{minipage}
\end{itemize}

\vspace{0.5 cm}
\begin{uncoverenv}<2->
\large And they have more resources:

\vspace{0.1 cm}
\begin{itemize}
\item more developers \hfill \begin{minipage}{0.7\linewidth}(trained in software engineering, not physics),\end{minipage}
\item open-source culture \hfill \begin{minipage}{0.7\linewidth}(mostly).\end{minipage}
\end{itemize}
\end{uncoverenv}
\end{frame}

\begin{frame}{The conclusion I take from this}
\vspace{0.5 cm}
\begin{center}
\LARGE We should be using their software.
\end{center}
\end{frame}

\begin{frame}{Our software developed \only<1>{outside}\only<2->{\xcancel{outside} \underline{before}} the big data ecosystem}
\vspace{0.5 cm}
\begin{uncoverenv}<4->
\fbox{\begin{minipage}{5 cm}
But that doesn't mean we're stuck. It's a challenge.
\end{minipage}}
\end{uncoverenv}

\vspace{-1.43 cm}
\only<1-2>{\includegraphics[width=\linewidth]{separation-1.png}}
\only<3->{\includegraphics[width=\linewidth]{separation-2.png}}
\end{frame}


\end{document}
